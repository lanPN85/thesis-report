\documentclass[a4paper,12pt]{report}
\usepackage{amsfonts}
\usepackage[numbers,sort&compress]{natbib}
\usepackage{comment}
\usepackage[utf8]{vietnam}
\usepackage{amssymb}
\usepackage{url}
\usepackage{subfig}
\usepackage{fancybox}
\usepackage{multirow}
\usepackage{multicol}
\usepackage{graphicx}
\usepackage{subfiles}
\usepackage[top=2cm, bottom=3cm, left=2cm, right=2cm]{geometry}
\usepackage[numbers]{natbib}
\usepackage{fancyhdr}
\usepackage{hyperref}
\usepackage{changepage}
\usepackage{framed}
\usepackage{multirow}
\usepackage{diagbox}
\usepackage{amsmath}
\usepackage{bm}
\usepackage{tabu}
\usepackage{booktabs}
\usepackage{listings}
\usepackage{placeins}
\usepackage{multirow}
\usepackage{listings}
\usepackage[table,xcdraw]{xcolor}
\usepackage{float}
\usepackage[justification=centering]{caption}
\usepackage{subfig}
\usepackage[english]{babel}
\usepackage{amssymb}
\usepackage{pifont}
\newcommand{\cmark}{\ding{51}}
\newcommand{\xmark}{\ding{55}}

\usepackage[linesnumbered,ruled,vlined]{algorithm2e}
\usepackage{algorithmic}
\SetAlFnt{\footnotesize}
\SetKw{KwDownTo}{downto}
\SetKw{KwTrue}{true}
\SetKw{KwFalse}{false}
\SetKwInOut{Input}{Input}
\SetKwInOut{Output}{Output}
\SetKw{KwAnd}{and}
\makeatletter
\newcommand{\nosemic}{\renewcommand{\@endalgocfline}{\relax}}
\newcommand{\dosemic}{\renewcommand{\@endalgocfline}{\algocf@endline}}
\newcommand{\pushline}{\Indp}
\newcommand{\popline}{\Indm\dosemic}
\let\oldnl\nl
\newcommand{\nonl}{\renewcommand{\nl}{\let\nl\oldnl}}
\makeatother

\hypersetup{
    colorlinks,
    citecolor=black,
    filecolor=black,
    linkcolor=blue,
    urlcolor=red 
}

\lstset{
    language=Python,
    numbers=none,
    tabsize=2,
    breaklines=true,
    basicstyle=\ttfamily\small,
    captionpos=t,
    stringstyle=\color{magenta},
    keywordstyle=\color{blue}\bfseries,
    numberstyle=\color{black}
}
\setlength{\parskip}{0.6em}

\makeatletter
\newcommand{\vast}{\bBigg@{4}}
\newcommand{\Vast}{\bBigg@{5}}
\newcommand{\vastl}{\mathopen\vast}
\newcommand{\vastm}{\mathrel\vast}
\newcommand{\vastr}{\mathclose\vast}
\newcommand{\Vastl}{\mathopen\Vast}
\newcommand{\Vastm}{\mathrel\Vast}
\newcommand{\Vastr}{\mathclose\Vast}
\makeatother

\renewcommand{\footrulewidth}{0.4pt}
\newcommand{\bigCI}{\mathrel{\text{\scalebox{1.07}{$\perp\mkern-10mu\perp$}}}}
\renewcommand{\baselinestretch}{1.2}

\hypersetup{
    colorlinks=true,
    linkcolor=blue,
    filecolor=magenta,      
    urlcolor=cyan,
}

% \setcounter{page}{3}
\hypersetup{colorlinks=true,urlcolor=black,linkcolor=black}
\graphicspath{ {images/} }
\lhead{}
\chead{}
\rhead{}

\title{BK.Synapse: Distributed Training of Deep Neural Networks for Object Detection}
\author{Phan Ngoc Lan}

\begin{document}

% \maketitle
\pagestyle{empty}
\thisfancypage{
\setlength{\fboxrule}{1pt}
\doublebox}{}

\begin{center}
    {\fontsize{12}{19}\fontfamily{cmr}\selectfont MINISTRY OF EDUCATION\\
     \textbf{HANOI UNIVERSITY OF SCIENCE AND TECHNOLOGY}}\\
     \begin{tabular}{lll}
    
    &   &                   \\
    &   &                   \\
    &   &                   \\
    &   &                   \\
    &   &                   \\
    &   &                   \\
    
    \end{tabular}
    
    \includegraphics[width=0.2\textwidth]{hust.jpeg}\\[1.3cm]
    {\fontsize{12}{43}\fontfamily{cmr}\selectfont \textbf{PHAN NGỌC LÂN} }\\[0.1cm]
     \begin{tabular}{lll}
    
    &   &                   \\
    &   &                   \\
    
    \end{tabular}

    {\fontsize{12}{10}\fontfamily{cmr}\fontseries{b}\selectfont \textbf{GRADUATION THESIS} }\\[0.1cm]

    \begin{tabular}{lll}
    &   &                   \\
    \end{tabular}
    
    {\fontsize{14}{18}\fontfamily{cmr}\selectfont \textbf{BK.Synapse: Distributed Deep Neural Network Training\\ for Object Detection} }\\[0.9cm]
    
    \begin{table}[ht]
    \centering
    \label{my-label}
    \begin{tabular}{lll}
                   
    \textbf{Instructor:} & Dr. ĐINH VIẾT SANG\\
    
    \end{tabular}
    \end{table} 
    \vspace{4cm}
    \fontsize{12}{19}\fontfamily{cmr}\selectfont Hanoi, 5 - 2019
\end{center}
\pagebreak

\selectlanguage{english}
\fontsize {13pt}{16pt}
\selectfont

\pagenumbering{gobble}
\chapter*{Acknowledgments}

\pagebreak

\chapter*{Abstract}
Training deep neural networks efficiently has been a thoroughly-researched topic over the history of deep learning. However, training at scale usually means adding on multiple layers of complex deployment logic and parallelization concerns, distracting researchers from the core of their algorithms. This thesis presents a framework called BK.Synapse that can facilitate distributed training while maintaining clarity, simplicity, and user-friendliness. The design is modular, allowing flexible and easy deployment on a variety of hardware specifications. We benchmark BK.Synapse in a case study: training a neural network for the object detection problem. Our results show a good amount of speedup over conventional training, with very few modifications to the existing codebase.
\pagebreak

\tableofcontents
\listoffigures
\listoftables
\pagebreak

\setcounter{page}{1}
\pagestyle{fancy}
\lfoot{Phan Ngoc Lan}
\cfoot{}
\rfoot{\thepage}
\pagenumbering{arabic}

\subfile{problem.tex}
\subfile{theory.tex}
\pagebreak
\subfile{propose.tex}
\subfile{results.tex}

\chapter{Conclusions}
This thesis has presented BK.Synapse, a framework for distributed neural network training. Each component has been described in detail, with heavy emphasis on user workflow and integration with existing code. We also showcased the tool with a case study on RetinaNet for object detection, and received positive, if somewhat early-stage, results.

At the same time, there is much room for improving BK.Synapse, which motivates our future development efforts. Our future works include:
\begin{itemize}
    \item Improving security and multi-user workflow, with the goal of having a viable public deployment.
    \item Adding support for the planned deep learning frameworks (Keras and TensorFlow).
    \item Adding more features in terms of monitoring and customizing the training process.
\end{itemize}

\pagebreak
\bibliographystyle{plain}
\bibliography{thesis}
\end{document}
