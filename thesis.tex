\documentclass[a4paper, 12pt, oneside]{report}
\usepackage[utf8]{inputenc}
\usepackage{amsfonts}
\usepackage[numbers,sort&compress]{natbib}
\usepackage{comment}
\usepackage[utf8]{vietnam}
\usepackage{amssymb}
\usepackage{url}
\usepackage{subfig}
\usepackage{fancybox}
\usepackage{multirow}
\usepackage{multicol}
\usepackage{graphicx}
\usepackage{subfiles}
\usepackage[left=3.50cm, right=2.00cm, top=2.00cm, bottom=2.00cm]{geometry}
\usepackage{fancyhdr}
\usepackage{hyperref}
\usepackage{changepage}
\usepackage{framed}
\usepackage{multirow}
\usepackage{diagbox}
\usepackage{amsmath}
\usepackage{bm}
\usepackage{tabu}
\usepackage{booktabs}
\usepackage{listings}
\usepackage{placeins}
\usepackage{multirow}
\usepackage{setspace}
% \usepackage[backend=biber, sorting=none]{biblatex}
\usepackage{listings}
\usepackage[nottoc]{tocbibind}
\usepackage[table,xcdraw]{xcolor}
\usepackage{float}
\floatstyle{plaintop}
\restylefloat{table}
\usepackage{caption}
\usepackage{subfig}
\usepackage[english]{babel}
\usepackage{amssymb}
\usepackage{pifont}
\usepackage[acronym, nonumberlist, shortcuts, toc]{glossaries}
\newcommand{\cmark}{\ding{51}}
\newcommand{\xmark}{\ding{55}}

\usepackage[linesnumbered,ruled,vlined]{algorithm2e}
\usepackage{algorithmic}
\SetAlFnt{\footnotesize}
\SetKw{KwDownTo}{downto}
\SetKw{KwTrue}{true}
\SetKw{KwFalse}{false}
\SetKwInOut{Input}{Input}
\SetKwInOut{Output}{Output}
\SetKw{KwAnd}{and}
\makeatletter
\newcommand{\nosemic}{\renewcommand{\@endalgocfline}{\relax}}
\newcommand{\dosemic}{\renewcommand{\@endalgocfline}{\algocf@endline}}
\newcommand{\pushline}{\Indp}
\newcommand{\popline}{\Indm\dosemic}
\let\oldnl\nl
\newcommand{\nonl}{\renewcommand{\nl}{\let\nl\oldnl}}
\makeatother

\hypersetup{
    colorlinks,
    citecolor=black,
    filecolor=black,
    linkcolor=blue,
    urlcolor=red 
}

\lstset{
    language=Python,
    numbers=none,
    tabsize=2,
    breaklines=true,
    basicstyle=\ttfamily\small,
    captionpos=t,
    stringstyle=\color{magenta},
    keywordstyle=\color{blue}\bfseries,
    numberstyle=\color{black}
}
\setlength{\parskip}{0.6em}

\makeatletter
\newcommand{\vast}{\bBigg@{4}}
\newcommand{\Vast}{\bBigg@{5}}
\newcommand{\vastl}{\mathopen\vast}
\newcommand{\vastm}{\mathrel\vast}
\newcommand{\vastr}{\mathclose\vast}
\newcommand{\Vastl}{\mathopen\Vast}
\newcommand{\Vastm}{\mathrel\Vast}
\newcommand{\Vastr}{\mathclose\Vast}
\makeatother

\renewcommand{\footrulewidth}{0.4pt}
\newcommand{\bigCI}{\mathrel{\text{\scalebox{1.07}{$\perp\mkern-10mu\perp$}}}}
\renewcommand{\baselinestretch}{1.2}

% \setcounter{page}{3}
\hypersetup{
    colorlinks=true,
    urlcolor=blue,
    linkcolor=black}
\graphicspath{ {images/} }
\lhead{}
\chead{}
\rhead{}

\makeglossaries
% \input{acronyms}
% \newglossaryentry{py}
{
    name={Python},
    description={An interpreted general-purpose programming language}
}

\loadglsentries[\acronymtype]{acronyms}
\loadglsentries{glossary}

\setlength{\parindent}{1.25cm}
\setlength{\parskip}{10pt}
\setlength{\columnsep}{0.5125cm}
\renewcommand{\baselinestretch}{1.2}

\fancypagestyle{IHA-fancy-style}{%
  \fancyhf{}% Clear header and footer
  \fancyhead[L]{\textit{This thesis is performed by: Phan Ngoc Lan - 2014505 - CNTT2-2 - K59}}
  \fancyfoot[R]{\thepage}% Custom footer
  \renewcommand{\headrulewidth}{0.4pt}% Line at the header visible
  \renewcommand{\footrulewidth}{0pt}% Line at the footer visible
}
% Redefine the plain page style
\fancypagestyle{plain}{%
  \fancyhf{}% Clear header and footer
  \fancyhead[L]{\textit{This thesis is performed by: Phan Ngoc Lan - 2014505 - CNTT2-2 - K59}}
  \fancyfoot[R]{\thepage}% Custom footer
  \renewcommand{\headrulewidth}{0.4pt}% Line at the header visible
  \renewcommand{\footrulewidth}{0pt}% Line at the footer visible
}
\pagestyle{IHA-fancy-style}

\begin{document}
\begin{spacing}{1.25}
    \thispagestyle{empty}
    \thisfancypage{\setlength{\fboxrule}{1pt}\doublebox}{}
    \begin{center}
        {\fontsize{17}{20}\selectfont HANOI UNIVERSITY OF SCIENCE AND TECHNOLOGY} \\
        {\fontsize{13}{17}\selectfont SCHOOL OF INFORMATION AND COMMUNICATION TECHNOLOGY} \\ [0.25cm]
        \textbf{---------------*---------------} \\ [1cm]
        \includegraphics[width=0.2\textwidth]{hust.jpeg} \\ [1cm]
        {\fontsize{25}{30}\selectfont \textbf{THESIS}} \\ [0.25cm]
        {\fontsize{14}{17}\selectfont SUBMITTED IN PARTIAL FULFILLMENT \\
        OF THE REQUIREMENTS FOR THE DEGREE OF} \\ [0.5cm]
        {\fontsize{25}{30}\selectfont \textbf{ENGINEER OF TECHNOLOGY}} \\ [0.5cm]
        {\fontsize{14}{17}\selectfont IN} \\ [0.5cm]
        {\fontsize{22}{26}\selectfont INFORMATION TECHNOLOGY} \\ [0.5cm]
        {\fontsize{15}{15}\selectfont \textbf{BK.SYNAPSE:\\DISTRIBUTED NEURAL NETWORK\\ TRAINING FRAMEWORK\\ AND ITS APPLICATION IN OBJECT DETECTION}}
        \\ [2.25cm]
        \begin{tabular}{ l l }
            Author & : \textbf{Phan Ngoc Lan} \\
            & : CNTT2-2 K59 \\
            & : 20142505 \\ [0.5cm]
            Supervisor & : Dr \textbf{Dinh Viet Sang}
        \end{tabular} \\ [2.25cm]
        {\fontsize{17}{20}\selectfont HANOI, 5 - 2019}
    \end{center}
\end{spacing}
\pagebreak

\selectlanguage{english}
\fontsize {13pt}{16pt}
\selectfont

\pagenumbering{roman}
\setcounter{page}{1}
\begin{spacing}{1.0}
    \chapter*{Requirements for the Thesis}
    \section*{Student Information}
    \begin{itemize}
        \begin{multicols}{2}
        \item \textbf{Full name:} PHAN NGOC LAN
        \item \textbf{Class:} CNTT2-2 K59
        \item \textbf{Tel:} 094 979 1149
        \item \textbf{Email:} lan.pn142505@sis.hust.edu.vn
        \item \textbf{Program:} Full-time program
        \end{multicols}
        \item \textbf{This thesis is performed at:} Department of Computer Science
        - School of Information and
        Communication Technology
        \item \textbf{This thesis is performed: from} 22/01/2019 \textbf{to}
        24/05/2019
    \end{itemize}
    \section*{Goals of the Thesis}
        \begin{itemize}
        \item Design and implement a distributed neural network training framework.
        \item Apply the framework in training a neural network for the object detection problem.
        \end{itemize}
    \section*{Main Tasks of the Thesis}
        \begin{itemize}
        \item Study the fundamentals of semantic segmentation problem in aerial
        photography and remote sensing.
        \item Study supervised learning algorithms in general and convolutional
        neural network in particular applied to computer vision.
        \item Propose fully convolutional networks used to solve 2D semantic
        labelling in aerial photography and remote sensing.
        \item Present experimental results and benchmark test results.
        \item Conclude and state future developments of the thesis.
        \end{itemize}
    \section*{Declaration of Student}
    I - Phan Ngoc Lan - hereby warrant that the work and presentation in this thesis are performed by myself under the supervision of \textit{Dr Dinh Viet Sang}.\\\\
    All results presented in this thesis are truthful and are not copied from any other works.\\\\
    \begin{minipage}{0.5\textwidth}
        \hfill
    \end{minipage}
    \begin{minipage}[t]{0.5\textwidth}
        \begin{center}
        \textit{Hanoi, 24th May, 2019\\Author\\[1cm]Phan Ngoc Lan}
        \end{center}
    \end{minipage}
    \subsection*{Attestation of the Supervisor on the Fulfillment of the
    Requirements for the Thesis:}
    \dotfill\\.\dotfill\\.\dotfill\\.\dotfill\\\\
    \begin{minipage}{0.5\textwidth}
        \hfill
    \end{minipage}
    \begin{minipage}[t]{0.5\textwidth}
        \begin{center}
        \textit{Hanoi, 24th May, 2019\\Supervisor\\[1cm]Dr Dinh Viet Sang}
        \end{center}
    \end{minipage}
\end{spacing}

\chapter*{Acknowledgments}

\pagebreak

\chapter*{Abstract}
Deep neural networks have seen many breakthroughs in the past several years, being applied to a wide variety of problems. Training these networks efficiently has been a thoroughly-researched topic throughout the history of deep learning. Major advancements have been made, including the use of multiple devices to further decrease training time. However, training at scale usually means adding on multiple layers of complex deployment logic and parallelization concerns, distracting researchers from the core of their algorithms. This thesis presents a framework called BK.Synapse that can facilitate distributed training while maintaining clarity, simplicity, and user-friendliness. The design is modular, allowing flexible and easy deployment on a variety of hardware specifications. We benchmark BK.Synapse in a case study: training a neural network for the object detection problem. Our results show a good amount of speedup over conventional training, with very few modifications to the existing codebase.

\chapter*{Tóm tắt nội dung đồ án}
Mạng neuron học sâu đã có rất nhiều đột phá mạnh mẽ trong những năm gần đây, và đã được áp dụng vào rất nhiều bài toán thực tế.

\chapter*{Summary}
The thesis is divided into 5 chapters:
\begin{itemize}
    \item Chapter 1 describes the problem of training neural networks at scale, as well as the object detection problem that would be used as a case study.
    \item Chapter 2 presents the theoretical background as well as related works, including machine learning, object detection and distributed training.
    \item Chapter 3 details the design and implementation of BK.Synapse, the proposed framework, and a case study where the framework is applied in the object detection problem.
    \item Chapter 4 reports on the results of several experiments on the framework.
    \item Chapter 5 presents our conclusions and lays out plans for future works.
\end{itemize}

\pagebreak
\pagenumbering{gobble}
\tableofcontents
\pagebreak
\pagenumbering{arabic}
\setcounter{page}{1}
\listoffigures
\listoftables

% \addcontentsline{toc}{chapter}{List of Abbreviations}
\printglossary[type=\acronymtype,style=long, title=List of Abbreviations]
% \addcontentsline{toc}{chapter}{Glossary}
\printglossary
\pagebreak

\subfile{problem.tex}
\subfile{theory.tex}
\pagebreak
\subfile{propose.tex}
\subfile{results.tex}

\chapter{Conclusions}
This thesis has presented BK.Synapse, a framework for distributed neural network training. Each component has been described in detail, with heavy emphasis on user workflow and integration with existing code. We also showcased the tool with a case study on RetinaNet for object detection, and received positive, if somewhat early-stage, results.

At the same time, there is much room for improving BK.Synapse, which motivates our future development efforts. Our future works include:
\begin{itemize}
    \item Improving security and multi-user workflow, with the goal of having a viable public deployment.
    \item Adding support for the planned deep learning frameworks (Keras and TensorFlow).
    \item Adding more features in terms of monitoring and customizing the training process.
\end{itemize}

\pagebreak
\bibliographystyle{plain}
\bibliography{thesis}
\end{document}
